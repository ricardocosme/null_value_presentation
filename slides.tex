\documentclass[t]{beamer}
% \usetheme{Luebeck}
% \usetheme{Malmoe}
\usetheme{default}

\usepackage{hyperref}
\usepackage[utf8]{inputenc} % this is needed for german umlauts
\usepackage[english]{babel} % this is needed for german umlauts
\usepackage[T1]{fontenc}    % this is needed for correct output 
                            % of umlauts in pdf
\usepackage{xcolor, colortbl}
\usepackage{listings}

\setbeamersize{text margin left=10pt}

\title{Valores nulos em containers}
\subtitle{Uma solução genérica para usar sentinelas como valores nulos}
\subject{C++}
\date{}

\begin{document}

  \lstset{language=C++,
    basicstyle=\small,
    keywordstyle=\color{blue}\ttfamily,
    stringstyle=\color{red}\ttfamily,
    commentstyle=\color{gray}\ttfamily,
    morecomment=[l][\color{magenta}]{\#}
  }

\begin{frame}
  \titlepage
\end{frame}

\begin{frame}[fragile]
  \frametitle{Problema}
  Como expressar nulos em um container de valores escalares?

  \begin{onlyenv}<1-2>
  \begin{lstlisting}[escapeinside=`']
    vector<double> {
      30.0,
      `\only<1>{\alert{?? /*null*/},}'`\only<2>{\alert{numeric\_limits::quiet\_NaN()},}'
      33.5
    };
  \end{lstlisting}
  \end{onlyenv}

  \begin{onlyenv}<2->
    \begin{itemize}
    \item{Uso de NaN(not a number)}
    \end{itemize}
  \end{onlyenv}

\end{frame}

\begin{frame}[fragile]
  \frametitle{Dificuldades}
  Tipo do valor não expressa a possibilidade de ausência de valor:
  \begin{lstlisting}[escapeinside=`']
    vector<`\alert{double}'> {
      30.0,
      numeric_limits<double>::quiet_NaN(),
      33.5
    };
  \end{lstlisting}

  % \begin{itemize}
  % \item{Tipo do valor não expressa nulidade}
  % \end{itemize}
\end{frame}

\begin{frame}[fragile]
  \frametitle{Dificuldades}
  Programação pouco segura para manipular o valor:
  \begin{lstlisting}[escapeinside=`']
    auto vec = vector<double> {
      30.0,
      numeric_limits<double>::quiet_NaN(),
      33.5
    };
  \end{lstlisting}

  \pause
    
  \begin{lstlisting}[escapeinside=`']
    //precondition: value eh um valor valido
    void use_value(double value) { /*...*/ }
  \end{lstlisting}

  \pause

  \begin{onlyenv}<3-4>
  \begin{lstlisting}[escapeinside=`']
    use_value(vec[i]); `\only<4->{\alert{//Bug em runtime}}'
  \end{lstlisting}
  \end{onlyenv}

  \begin{onlyenv}<5->
  \begin{lstlisting}[escapeinside=`']
    if (`\alert{!isnan(vec[i])}')
      use_value(vec[i]);
  \end{lstlisting}
  \end{onlyenv}

  \begin{itemize}
  \item<6->{O compilador não me ajuda, é necessário lembrar uso de isnan()}
  \end{itemize}
\end{frame}

\begin{frame}[fragile]
  \frametitle{Dificuldades}
  \begin{itemize}
    \item<1->{Não existe NaN para integrais}
    \item<2->{É necessário escolher um valor válido do conjunto de valores}
    \item<3->{O valor escolhido pode não funcionar para todos os contextos do software}
  \end{itemize}

  \begin{visibleenv}<4->
  \begin{lstlisting}[escapeinside=`']
    auto vec_1 = vector<unsigned char> {
      145,
      numeric_limits<unsigned char>::max(),
      50
    };
  \end{lstlisting}
  \end{visibleenv}

  \begin{visibleenv}<5->
  \begin{lstlisting}[escapeinside=`']
    auto vec_2 = vector<unsigned char> {
      255,
      numeric_limits<unsigned char>::min(),
      89
    };
  \end{lstlisting}
  \end{visibleenv}
\end{frame}

\begin{frame}[fragile]
  \frametitle{Dificuldades}
  Posso misturar valores nulos diferentes de containers do mesmo tipo:
  \begin{lstlisting}[escapeinside=`']
    //precondition: value eh um valor valido
    void use_value(unsigned char value) { /*...*/ }
  \end{lstlisting}

  \begin{onlyenv}<1>
  \begin{lstlisting}
    auto null = numeric_limits<unsigned char>::max();
    if (vec_1[i] != null)
      use_value(vec_1[i])
  \end{lstlisting}
  \end{onlyenv}

  \begin{onlyenv}<2->
  \begin{lstlisting}[escapeinside=`']
    auto null = numeric_limits<unsigned char>::`\alert{min}'();
    if (vec_1[i] != null)
      use_value(vec_1[i]) `\only<3->{\alert{//Bug em runtime}}'
  \end{lstlisting}
  \end{onlyenv}
\end{frame}

\begin{frame}[fragile]
  \frametitle{Dificuldades}
  \begin{itemize}
    \item<1->{Baixa expressividade}
    \item<2->{O compilador não ajuda a verificar nulos}
    \item<3->{Programador pode misturar valores nulos}
  \end{itemize}
\end{frame}

\begin{frame}[fragile]
  \frametitle{Boost.Optional}
  \begin{itemize}
    \item<1->{Expressa no tipo a possibilidade de ausência de valor}
  \end{itemize}
  \begin{lstlisting}[escapeinside=`']
    auto vec_1 = vector<boost::optional<double>> {
      30.0,
      {},
      33.5
    };
  \end{lstlisting}

  % \pause

  % \begin{lstlisting}[escapeinside=`']
  %   //precondition: value eh um valor valido
  %   void use_value(double value) { /*...*/ }
  % \end{lstlisting}

  \begin{onlyenv}<2-3>
  \begin{lstlisting}[escapeinside=`']
    use_value(vec_1[i]); `\visible<3>{\alert{//Ótimo. Erro de compilação.}}'
  \end{lstlisting}
  \end{onlyenv}

  \begin{itemize}
    \item<3>{Compilador notifica que o nulo deve ser tratado}
  \end{itemize}

  \begin{onlyenv}<4->
  \begin{lstlisting}[escapeinside=`']
    if (vec_1[i]) use_value(*vec_1[i]);
  \end{lstlisting}
  \end{onlyenv}

  \begin{itemize}
    \item<5->{Não é necessário lidar com valores especiais(NaN, INT\_MAX, ...)}
  \end{itemize}
\end{frame}

\begin{frame}[fragile]
  \frametitle{Boost.Optional}
  \begin{itemize}
    \item<1->{Ineficiência(no espaço) inviável para container com muitos elementos}
  \end{itemize}
  \begin{lstlisting}[escapeinside=`']
    sizeof(boost::optional<double>>) = 2*sizeof(double>)
  \end{lstlisting}
  \begin{itemize}
    \item<2->{Se double é 8 bytes então boost::optional<double> é 16 bytes!}
    \item<3->{Compilador alinha a memória da estrutura definida, por exemplo:}
  \end{itemize}

  \begin{onlyenv}<4-6>
  \begin{lstlisting}[escapeinside=`']
    struct optional_double {
      double value; `\only<5->{\alert{//8 bytes}}'
      bool _initialized; `\only<6->{\alert{//1 byte}}' 
    };
  \end{lstlisting}
  \end{onlyenv}

  \begin{onlyenv}<7->
  \begin{lstlisting}[escapeinside=`']
    struct optional_double {
      double value; `\alert{//8 bytes}'
      bool _initialized; `\alert{//1 byte}' 
      `\alert{char \_pad[7]; //7 bytes}'
    };
  \end{lstlisting}
  \end{onlyenv}
\end{frame}

\begin{frame}[fragile]
  \frametitle{Nullable(ak\_toolkit::markable)}
  Um ``optional'' com valor nulo definido em compile time
  \begin{itemize}
    \item<1->{Nullable é somente uma conveniência para ak\_toolkit::markable}
      \begin{itemize}
      \item<2->{Tentativa de um nome mais expressivo}
      \item<3->{Define valores nulos padrões para integrais e ponto flutuante: numeric\_limits<Int>:max() e NaN.}
      \end{itemize}
    \item<4->{Expressividade do boost::optional}
    \item<5->{Segurança no uso como no boost::optional}
    \item<6->{Eficiente: sizeof(Nullable<T>) == sizeof(T)}
  \end{itemize}

  \begin{visibleenv}<7->
  \begin{lstlisting}[escapeinside=`']
    auto vec = vector<Nullable<double>> {
      30.0,
      Nullable<double>{},
      33.5
    };
  \end{lstlisting}
  \end{visibleenv}

  \begin{visibleenv}<8->
  \begin{lstlisting}[escapeinside=`']
    if (vec[i].has_value()) 
      use_value(vec[i].value());
  \end{lstlisting}
  \end{visibleenv}

\end{frame}

\begin{frame}[fragile]
  \frametitle{Construção com nulos definidos em runtime}
  Construção do container a partir de fonte externa que utiliza valor nulo definido em runtime
  \begin{itemize}
    \item<1->{Não sabemos qual é o valor nulo quando escrevemos o software}
    \item<2->{Após importação o valor nulo em runtime não é mais necessário para o container}
    \item<3->{Conveniência toNullable(value, null) para construir Nullable<T>}
      \begin{itemize}
      \item<4->{Evita código boilerplate de verificação de nulo}
      \end{itemize}
  \end{itemize}


  \begin{onlyenv}<5->
  \begin{lstlisting}[escapeinside=`']
    (null == value) ? Nullable<T>() : Nullable<T>(value)
  \end{lstlisting}
  \end{onlyenv}

  \begin{onlyenv}<6->
  \begin{lstlisting}[escapeinside=`']
    double null = -999.25;
    vec.push_back(toNullable(value, null));
  \end{lstlisting}
  \end{onlyenv}
\end{frame}

\begin{frame}[fragile]
  \frametitle{[Extra] Sentinelas customizadas (MarkPolicy)}
  Uso de uma policy customizada MarkPolicy para usar um valor de sentinela customizado:
  \begin{itemize}
    \item<1->{Permite por exemplo definir valores de mesmo tipo com valores nulos distintos}
  \end{itemize}

  \begin{onlyenv}<2->
  \begin{lstlisting}[basicstyle=\small]
    using NullableIntMin = Nullable<
      int, 
      mark_int<int, numeric_limits<int>::min()>
    >;

    using NullableIntMax = Nullable<
      int, 
      mark_int<int, numeric_limits<int>::max()>
    >;
  \end{lstlisting}
  \end{onlyenv}
\end{frame}

\begin{frame}[fragile]
  \frametitle{Referências}
  \begin{itemize}
    \item{Boost.Optional \url{http://www.boost.org/doc/libs/release/libs/optional}}
    \item{\url{https://github.com/akrzemi1/markable}}
  \end{itemize}
\end{frame}
\end{document}